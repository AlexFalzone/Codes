\documentclass{article}
\usepackage[utf8]{inputenc}
\usepackage{algorithm}
\usepackage{algorithmic}

\title{README.md}
\author{Alex Falzone }
\date{January 2021}

\begin{document}

\maketitle

\section{Introduction}
    Creata nel 1975 dall'informatico Peter van Emde Boas.\\
    Questo albero ha la particolare caratteristica di svolgere le operazioni di search, insert, delete, minimum, maximum, successor, predecessor nel tempo O(lg lg u) dove u rappresenta la dimensione dell'universo.\\
    Possiamo pensare a quest'albero come un array A (diviso in $\sqrt{u}$ parti) contenenti 0 o 1. 
    Definito un array \textbf{summary}[0 $\dots$ $\sqrt{u}$ - 1], dove summary[i] contiene 1 se e soltanto se il sottoarray A[ i$\sqrt{u}$ $\dots$ (i + 1)$\sqrt{u}$ - 1] contiene almeno un 1. Questo sottoarray di A di $\sqrt{u}$ bit è detto i-esimo.\\ \textbf{cluster}.


\section{Struttura dell'albero}
    Indichiamo con $\sqrt[\downarrow]{u}$ la radice quadrata \textbf{inferiore}, ovvero 
    \begin{equation}
        \centering2^{\lfloor lg(u) / 2 \rfloor}.
    \end{equation}
    Mentre con $\sqrt[\uparrow]{u}$ la radice \textbf{superiore}, ovvero 
    \begin{equation}
        \centering2^{\lceil lg(u) / 2 \rceil}
    \end{equation}
    Denotiamo con vEB(u) un albero vEB con dimensione dell'universo pari a \textbf{u} e, a meno che u non sia uguale alla dimensione base 2, l'attributo summary punta a un albero vEB($\sqrt[\uparrow]{u}$) e l'array cluster[ 0 $\dots$ $\sqrt[\uparrow]{u}$ - 1] punta ai $\sqrt[\uparrow]{u}$ alberi vEB($\sqrt[\downarrow]{u}$).\\
    Inoltre all'interno di un cluster è presente un valore \textbf{min} e un valore \textbf{max}. Banalmente memorizzano rispettivamente l'elemento minimo e massimo nell'albero vEB. Inoltre è importante notare che l'elemento memorizzato in \textbf{min} non appare in nessuno degli albero di ricorsione vEB($\sqrt[\downarrow]{u}$) cui punta l'array cluster (diversamente accade per \textbf{max}).\\
    Successivamente è necessario stabilire come accedere sia al numero di cluster di un determinato valore(x), sia alla posizione del valore(x) all'interno del cluster; Esso avviene usando rispettivamente:\\
    \begin{equation}
        high(x) = \lfloor x / \sqrt[\downarrow]{u} \rfloor
    \end{equation}
    \begin{equation}
        low(x) = x mod \sqrt[\downarrow]{u}
    \end{equation}
    \begin{equation}
        index(x, y) = (x\sqrt[\downarrow]{u}) + y
    \end{equation}
    Dove index rappresenta il numero(x) e la posizione(y) del cluster. 
    \\

    \subsection{Operazioni vEB}
        \subsubsection{Minimo e massimo}
            \begin{algorithm}
                \caption{Minimum(V)}
                    {return V.min}
            \end{algorithm}
           \begin{algorithm}
                \caption{Maximum(V)}
                    {return V.max}
            \end{algorithm}
            
            Poiché il massimo e il minimo sono memorizzati negli attribuiti \textbf{min} e \textbf{max} essi richiedono tempo costante, ovvero O(1).
        
        \subsubsection{Successore e Predecessore}
            
            
\end{document}
